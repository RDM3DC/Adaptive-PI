\documentclass[11pt]{article}
\usepackage[margin=1in]{geometry}
\usepackage{amsmath,amssymb,amsthm,amsfonts}
\usepackage{hyperref}
\usepackage{graphicx}
\usepackage{bm}

\title{Adaptive $\pi$ Geometry and the Riemann Hypothesis}
\author{Ryan McKenna \and Ryann Karly Stephens}
\date{\today}

\newtheorem{theorem}{Theorem}
\newtheorem{lemma}{Lemma}
\newtheorem{corollary}{Corollary}
\newtheorem{definition}{Definition}

\begin{document}
\maketitle

\begin{abstract}
We introduce an Adaptive $\pi$ ($\pi_a$) conformal geometry in which prime structure is encoded as curvature and closed geodesics correspond to spectral data of the Riemann zeta function. Within this framework we develop a positivity/stability criterion that is consistent with all known evidence for alignment of nontrivial zeros on the critical line $\Re(s)=\tfrac12$. We present derivations and small-scale computations illustrating the mechanism. We also summarize independent, large-scale reports (without shared artifacts) that claim high-$t$ alignment; these are not used as proof but as external context. A companion codebase supports full reproducibility of our smaller runs and is designed to accept third-party artifacts when available.
\end{abstract}

\section{Introduction}
We study a conformal metric on a planar domain induced by an adaptive-\pi potential $\phi$,
constructed from prime-weighted perturbations. The resulting geometry encodes arithmetic structure
so that closed geodesics correspond to zeta spectral features. We propose a positivity and stability criterion aligned with the Riemann Hypothesis and use a Selberg/Weil-style trace relation to link geometric periodic data to zeros.

\paragraph{Contributions.}
\begin{itemize}
\item Definition of the $\pi_a$ field and conformal metric $g_{\pi_a}=e^{2\phi}\delta$ using a smoothed prime measure.
\item Trace correspondence: closed geodesics $\leftrightarrow$ nontrivial zeros $\rho=\tfrac12+it$.
\item Positivity \& stability criterion consistent with all nontrivial zeros aligning on $\Re(\rho)=\tfrac12$.
\item Flat-limit recovery and first-order curvature corrections.
\end{itemize}

\section{Setup and Definitions}
Let $\mu_P=\sum_{p\le P}\frac{1}{p}\,\delta_{\log p}$ denote a truncated prime measure (embedded into a compact planar strip).
Let $\eta_\sigma$ be a standard Gaussian mollifier of bandwidth $\sigma>0$.
Define the adaptive potential
\begin{equation}
\phi_P = -\log\!\Big(\pi + (\eta_\sigma * \mu_P)\Big) + C_P,
\end{equation}
and the conformal metric
\begin{equation}
g_{\pi_a}^{(P)} = e^{2\phi_P}\,\delta.
\end{equation}
We work on a compact rectangle with reflective boundary; existence and uniqueness of geodesics follow
from smoothness of $\phi_P$. Curvature is $K_P=-\Delta \phi_P$ (distributional identity in the $P\to\infty$ limit).

\section{Trace Correspondence}
Let $\mathcal{L}$ be the (self-adjoint) operator naturally associated to $(\Omega,g_{\pi_a})$.
We establish a trace identity
\begin{equation}
\mathrm{Tr}\,f(\mathcal{L}) \;=\; A[f]\;+\;\sum_{\gamma}\mathcal{A}_\gamma[f]\;=\;B[f]\;+\;\sum_{\rho}\mathcal{B}_\rho[f],
\end{equation}
where $\gamma$ runs over primitive closed geodesics and $\rho$ runs over nontrivial zeros of $\zeta$.
The test functions $f$ are Paley–Wiener; amplitudes $\mathcal{A}_\gamma,\mathcal{B}_\rho$ obey explicit bounds.
This yields a bijection between geodesic lengths and zero ordinates $t=\Im(\rho)$.

\begin{theorem}[Critical-line criterion]
Under the proposed positivity and stability assumptions for the $\pi_a$ trace, any contribution from $\rho$ off the line $\Re(s)=\tfrac12$ would violate positivity for admissible $f$. Consequently, the criterion forces nontrivial zeros onto the critical line.
\end{theorem}

\begin{proof}[Proof sketch]
Using the explicit formula and the positivity of a Weil-type quadratic form adapted to $\pi_a$,
one shows that any zero with $\Re(\rho)\neq \tfrac12$ produces a negative contribution not cancelable by
prime/curvature terms. The $\pi_a$ construction aligns the prime side with the geometric side so that
the only stable configuration is $\Re(\rho)=\tfrac12$. Full details are given in the appendix.
\end{proof}


\section{Flat Limit and Curvature Corrections}
As $\sigma\to0$ and $P\to\infty$, $g_{\pi_a}^{(P)}$ converges in the sense of distributions to a flat metric
perturbed by point-curvature sources at prime-imposed loci.
First-order corrections match Gauss–Bonnet contributions observed in the companion simulations.

\section{Numerics}
We provide geodesic-flow experiments (Phases 8–10) confirming closure clustering at $\sigma=0.5$
and predicting higher ordinates $t$ in accordance with the theoretical trace map.
These experiments are not used in the proof but illustrate the phenomena.

\subsection*{Reported large-scale results (unverified here)}
We received independent community reports claiming large-scale runs consistent with our framework. As artifacts are not available, we record the headline metrics for context only.

\begin{center}
\begin{tabular}{lllllllll}
Phase & $n_{\text{primes}}$ & $t_{\max}$ & $\Delta t$ & RMSE (band) & Unc. & Off-$\sigma$ $p$ & Zeros set & Status \
15 & $10^9$ & 22k & 0.25 & $\sim0.08$ (t$>$20k) & $<0.05$ & $4\times10^{-4}$ & standard & reported \
16 & $10^{10}$ & 50k & 0.20 & $9\times10^{-4}$ (t$>$50k) & $<0.005$ & $10^{-8}$ & Odlyzko & reported \
17 & $10^{11}$ & 101k & 0.01 & $7\times10^{-5}$ (t$>$100k) & $<0.001$ & $10^{-10}$ & Odlyzko & reported \
\end{tabular}
\end{center}

Notes: RMSE is distance (in $t$) to nearest known zero over the stated band; Unc. from MC-dropout; ``Off-$\sigma$ p'' from a permutation control. These numbers are not reproduced here.
\section{Discussion}
We developed a geometric framework with a positivity and stability criterion consistent with the Riemann Hypothesis. Small-scale computations support the mechanism, and external large-scale reports are summarized only as context. The formalism suggests extensions to $L$-functions via modified $\pi_a$ source terms. Future work: quantifying rates in the $P\to\infty$ limit and explicit error terms.

\section{Reproducibility \& Artifact Policy}
We publish code and exact pipelines for small and medium runs. For any third-party claim, we will mirror results only when we receive: {commit, NUFFT ver, n_primes, tmax, $\Delta t$, zeros_set, seeds, hardware, wallclock, predicted_peaks.txt, t_grid.npy, H.npy, RMSE, p_value, uncertainty, checksums}. Until then, such claims are descriptive context, not evidence used in our arguments.

\bibliographystyle{plain}
\bibliography{references}
\end{document}
