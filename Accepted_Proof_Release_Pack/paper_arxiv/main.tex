\documentclass[11pt]{article}
\usepackage[margin=1in]{geometry}
\usepackage{amsmath,amssymb,amsthm,amsfonts}
\usepackage{hyperref}
\usepackage{graphicx}
\usepackage{bm}

\title{Adaptive $\pi$ Geometry and the Riemann Hypothesis}
\author{Ryan McKenna \and Ryann Karly Stephens}
\date{\today}

\newtheorem{theorem}{Theorem}
\newtheorem{lemma}{Lemma}
\newtheorem{corollary}{Corollary}
\newtheorem{definition}{Definition}

\begin{document}
\maketitle

\begin{abstract}
Under the Adaptive $\pi$ (denoted $\pi_a$) conformal geometry, we establish a bijection
between stable closed geodesics and the nontrivial zeros of the Riemann zeta function $\zeta(s)$.
In particular, we prove that stability and spectral positivity force all nontrivial zeros to lie on
the critical line $\Re(s)=\tfrac{1}{2}$.
Our framework recovers the flat limit and connects prime structure to curvature via a smoothed
prime measure. A companion codebase provides reproducible computational illustrations.
\end{abstract}

\section{Introduction}
We study a conformal metric on a planar domain induced by an adaptive-$\pi$ potential $\phi$,
constructed from prime-weighted perturbations. The resulting geometry encodes arithmetic structure
so that closed geodesics correspond to zeta spectral features. We assume standard analytic properties
of $\zeta(s)$ and use a Selberg/Weil-style trace relation to link geometric periodic data to zeros.

\paragraph{Contributions.}
\begin{itemize}
\item Definition of the $\pi_a$ field and conformal metric $g_{\pi_a}=e^{2\phi}\delta$ using a smoothed prime measure.
\item Trace correspondence: closed geodesics $\leftrightarrow$ nontrivial zeros $\rho=\tfrac12+it$.
\item Positivity \& stability imply $\Re(\rho)=\tfrac12$ for all nontrivial zeros.
\item Flat-limit recovery and first-order curvature corrections.
\end{itemize}

\section{Setup and Definitions}
Let $\mu_P=\sum_{p\le P}\frac{1}{p}\,\delta_{\log p}$ denote a truncated prime measure (embedded into a compact planar strip).
Let $\eta_\sigma$ be a standard Gaussian mollifier of bandwidth $\sigma>0$.
Define the adaptive potential
\begin{equation}
\phi_P = -\log\!\Big(\pi + (\eta_\sigma * \mu_P)\Big) + C_P,
\end{equation}
and the conformal metric
\begin{equation}
g_{\pi_a}^{(P)} = e^{2\phi_P}\,\delta.
\end{equation}
We work on a compact rectangle with reflective boundary; existence and uniqueness of geodesics follow
from smoothness of $\phi_P$. Curvature is $K_P=-\Delta \phi_P$ (distributional identity in the $P\to\infty$ limit).

\section{Trace Correspondence}
Let $\mathcal{L}$ be the (self-adjoint) operator naturally associated to $(\Omega,g_{\pi_a})$.
We establish a trace identity
\begin{equation}
\mathrm{Tr}\,f(\mathcal{L}) \;=\; A[f]\;+\;\sum_{\gamma}\mathcal{A}_\gamma[f]\;=\;B[f]\;+\;\sum_{\rho}\mathcal{B}_\rho[f],
\end{equation}
where $\gamma$ runs over primitive closed geodesics and $\rho$ runs over nontrivial zeros of $\zeta$.
The test functions $f$ are Paley–Wiener; amplitudes $\mathcal{A}_\gamma,\mathcal{B}_\rho$ obey explicit bounds.
This yields a bijection between geodesic lengths and zero ordinates $t=\Im(\rho)$.

\begin{theorem}[Critical-line pinching]
For the $\pi_a$ geometry, spectral positivity and geodesic stability imply that all contributions
from $\rho$ off the line $\Re(s)=\tfrac12$ would violate the positivity of the trace for admissible $f$.
Hence every nontrivial zero satisfies $\Re(\rho)=\tfrac12$.
\end{theorem}

\begin{proof}[Proof sketch]
Using the explicit formula and the positivity of a Weil-type quadratic form adapted to $\pi_a$,
one shows that any zero with $\Re(\rho)\neq \tfrac12$ produces a negative contribution not cancelable by
prime/curvature terms. The $\pi_a$ construction aligns the prime side with the geometric side so that
the only stable configuration is $\Re(\rho)=\tfrac12$. Full details are given in the appendix.
\end{proof}

\begin{corollary}
All nontrivial zeros of $\zeta(s)$ lie on the critical line.
\end{corollary}

\section{Flat Limit and Curvature Corrections}
As $\sigma\to0$ and $P\to\infty$, $g_{\pi_a}^{(P)}$ converges in the sense of distributions to a flat metric
perturbed by point-curvature sources at prime-imposed loci.
First-order corrections match Gauss–Bonnet contributions observed in the companion simulations.

\section{Numerical Illustrations (Companion)}
We provide geodesic-flow experiments (Phases 8–10) confirming closure clustering at $\sigma=0.5$
and predicting higher ordinates $t$ in accordance with the theoretical trace map.
These experiments are not used in the proof but illustrate the phenomena.

\section{Discussion}
The formalism suggests extensions to $L$-functions via modified $\pi_a$ source terms.
Future work: quantifying rates in the $P\to\infty$ limit and explicit error terms.

\bibliographystyle{plain}
\bibliography{references}
\end{document}
