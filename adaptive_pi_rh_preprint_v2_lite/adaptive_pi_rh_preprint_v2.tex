
\documentclass[11pt]{article}
\usepackage{amsmath,amssymb,amsthm}
\usepackage{graphicx}
\usepackage{hyperref}
\usepackage[margin=1in]{geometry}
\usepackage{microtype}
\usepackage{mathpazo}

\newtheorem{theorem}{Theorem}
\newtheorem{lemma}{Lemma}
\newtheorem{proposition}{Proposition}
\newtheorem{assumption}{Assumption}
\newtheorem{definition}{Definition}

\title{Adaptive $\pi$ Geometry, Geodesic Resonance, and a Conditional Route Toward the Riemann Hypothesis}
\author{Ryan McKenna \and Ryann Karly Stephens}
\date{2025-08-18}

\begin{document}
\maketitle

\begin{abstract}
We develop a geometric framework in which primes generate an adaptive curvature field $\pi_a$ on a two-dimensional domain; geodesics in the conformal metric $g=e^{2\phi}\delta$ (with $\phi \propto \pi_a-\pi$) display closure stability preferentially near the critical line $\Re(s)=\tfrac12$. We formalize a set of assumptions under which the gradient of the conformal factor aligns with the gradient of $\Re\log\zeta(\sigma+it)$ and state a conditional stability theorem: closed orbits arise only on $\sigma=\tfrac12$. We report numerical ablations supporting the stability dip near the critical line and outline a program to remove tuning and pass to the limit in prime truncation and kernel width.
\end{abstract}

\section{Setup}
Let $\mathcal{P}_P=\{p_1,\dots,p_P\}$ be the first $P$ primes and map each to the unit square via
\begin{equation}
(x_p,y_p) = \Big(\frac{p}{p_{\max}}, \frac{\log p}{\log p_{\max}}\Big).
\end{equation}
For a radial kernel $K_\varepsilon(r)=\exp(-r^2/(2\varepsilon^2))$ define the adaptive field
\begin{equation}
\pi_a^{(P,\varepsilon)}(x,y;\sigma) \;=\; \pi + \sum_{p\in \mathcal{P}_P} \frac{1}{p\,(\,|\sigma-\tfrac12|+\eta)}\, K_\varepsilon\!\Big(\sqrt{(x-x_p)^2 + (y-y_p)^2}\Big),
\end{equation}
with small $\eta>0$ used only for numerical stability. Set the conformal potential
\begin{equation}
\phi^{(P,\varepsilon)}(x,y;\sigma) \;=\; \alpha\,\frac{\pi_a^{(P,\varepsilon)}(x,y;\sigma)-\pi}{\|\pi_a^{(P,\varepsilon)}(\cdot,\cdot;\sigma)-\pi\|_{\infty}},
\qquad g^{(P,\varepsilon)} \;=\; e^{2\phi^{(P,\varepsilon)}}\,\delta.
\end{equation}
Geodesics $x(t)\in[0,1]^2$ for $g^{(P,\varepsilon)}$ satisfy
\begin{equation}
\ddot x \;=\; -2\big(\nabla\phi\cdot \dot x\big)\dot x + (\nabla\phi)\,\|\dot x\|^2.
\end{equation}
We say a trajectory is \emph{closed} if $x(T)\approx x(0)$ within a tolerance after a warm-up window; the \emph{closure error} is $\min_{t\ge t_0}\|x(t)-x(0)\|$.

\subsection*{Assumptions for the limit model}
We isolate the analytic hypotheses we seek to verify.
\begin{assumption}[Kernel approximation]
As $P\to\infty$ and $\varepsilon\to 0$ with $P\varepsilon^2\to\infty$, the field $\pi_a^{(P,\varepsilon)}-\pi$ converges (in the sense of distributions) to a measure supported on $\{(x_p,y_p)\}$ whose $y$-marginal approximates the prime log spectrum.
\end{assumption}
\begin{assumption}[Gradient--Euler alignment]
For fixed $\sigma\in(0,1)$ and $t\in\mathbb{R}_+$ identified with $y$ via $t=\Lambda y$, there exists a scaling $\beta(\sigma,\varepsilon,P)$ such that
\begin{equation}
\nabla\phi^{(P,\varepsilon)}(\sigma,y) \;\xrightarrow[P\to\infty,\ \varepsilon\to 0]{}\; \beta(\sigma)\,\nabla_{(\sigma,t)}\,\Re\log\zeta(\sigma+it),
\end{equation}
uniformly on compact subsets away from poles/zeros.
\end{assumption}
\begin{assumption}[Symmetry]
The limit inherits the functional equation symmetry so that the lateral (in $\sigma$) odd part of the drift cancels at $\sigma=\tfrac12$.
\end{assumption}

\section{Formal Statements}
\begin{proposition}[Flow--Euler link (heuristic)]
Under Assumptions 1--2, the stationary points of the $\pi_a$-induced flow along the line $x=\sigma$ correspond to critical points of $\Re\log\zeta(\sigma+it)$, hence the geodesic descent equilibria approximate level-set tangencies of $\Re\zeta$.
\end{proposition}

\begin{lemma}[Asymmetry drift away from the critical line (heuristic)]
Assume 1--3. For $\sigma\neq \tfrac12$, the lateral component of $\nabla\phi$ possesses a nonzero mean sign inherited from the asymmetric part of $\Re\log\zeta(\sigma+it)$, inducing secular drift that destabilizes closed orbits. At $\sigma=\tfrac12$ the mean lateral drift cancels.
\end{lemma}

\begin{theorem}[Critical-line closure stability (conditional)]
Assume 1--3 and that the geodesic flow generated by $g=e^{2\phi}\delta$ admits a compact global attractor of bounded complexity. Then any family of closed orbits persisting in the limit $P\to\infty,\ \varepsilon\to 0$ must lie on $\sigma=\tfrac12$. Moreover, closure centers project (via $t=\Lambda y$) to a discrete set asymptotically aligned with the imaginary parts of zeros of $\zeta$.
\end{theorem}

\paragraph{Remarks.} The theorem is conditional: validating Assumptions 1--3 and the attractor hypothesis is part of the open program. The statements formalize what our numerics suggest: a stability dip at $\sigma=\tfrac12$ and closure clustering near known zero heights.

\section{Numerical Ablations}
We report four ablations; figures referenced are included in the bundle.
\begin{itemize}
\item \textbf{(A) $\sigma$-sweep closures.} Mean closure error vs.\ $\sigma$ exhibits a dip near $\tfrac12$ (Figure~\ref{fig:closure}).
\item \textbf{(B) Field and geodesics at $\sigma=\tfrac12$.} Low-$\phi$ channels host many near-closed trajectories (Figure~\ref{fig:field_geo}).
\item \textbf{(C) Closure heatmap.} Regions of tight closure concentrate along filaments (Figure~\ref{fig:heatmap}).
\item \textbf{(D) Harmonic overlay.} Dips of $H(t)=\sum_{p\le P}\cos(2\pi \log p\, t)$ correlate with minima of a 1D $\pi_a$ line and align with the first known zeros (overlay figure).
\end{itemize}

\begin{figure}[h!]
  \centering
  \includegraphics[width=.8\linewidth]{phase2p5_closure_vs_sigma.png}
  \caption{Closure stability vs $\sigma=\Re(s)$ (Phase II.5). A pronounced minimum appears near $\sigma=\tfrac12$.}
  \label{fig:closure}
\end{figure}

\begin{figure}[h!]
  \centering
  \includegraphics[width=.45\linewidth]{phase2p5_field_sigma05.png}\hfill
  \includegraphics[width=.45\linewidth]{phase2p5_geodesics_sigma05.png}
  \caption{$\phi$ field and geodesic trajectories at $\sigma=\tfrac12$.}
  \label{fig:field_geo}
\end{figure}

\begin{figure}[h!]
  \centering
  \includegraphics[width=.7\linewidth]{phase2p5_closure_heatmap_sigma05.png}
  \caption{Closure error heatmap at $\sigma=\tfrac12$. Darker is tighter closure.}
  \label{fig:heatmap}
\end{figure}

\section{Program Toward Rigor}
\paragraph{Remove tuning.} Prove that the stability minimum persists without the explicit $1/|\sigma-\tfrac12|$ factor and for general kernels in a compact class; pass to $P\to\infty$, $\varepsilon\to 0$ with quantitative error bounds.
\paragraph{Connect to explicit formula.} Identify a functional $\mathcal{F}[\phi;\sigma,t]$ whose first variation reproduces the smoothed explicit formula and show that $\delta \mathcal{F}=0$ coincides with the flow equilibria.
\paragraph{Spectral angle.} Pursue a Hilbert--Pólya surrogate by constructing a self-adjoint operator whose ray dynamics approximate the geodesic flow, then compare spectra.
\paragraph{Equivalence target.} Aim for an equivalence with a known RH criterion (Weil or Li) expressed through the $\pi_a$ potential.

\section*{Data \& Code}
All scripts and figures accompany this preprint. See the Phase II.5 simulator and Phase 5 prototypes described in the repository documentation.

\end{document}
